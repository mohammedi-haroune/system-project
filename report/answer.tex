\subsection{Question 1}
\subsubsection{Implementaion}
Pour l'implémentation du problème donnée en utilisant le système \texttt{IPC V},
on a définit les sémaphores indiqué dans la solution théorique donnée à l'aide
des appels système  linux \texttt{semget, semctl} et une mémoire partagée pour
sauvegarder le nombre de clients embarquées et débarquées (la structure
\texttt{state}). les mémoires partagées sont gérées à l'aide des appels système 
\texttt{shmget, shmat}. Notant que l'utlisation de ces appels système nécessite la création d'un clé \texttt{key\_t}.

Pour facilité la tache de compilation des différents fichiers \textbf{C} on a
utlisé l'outil \textbf{CMake}. pour cela on écrit le fichier
\texttt{CMakeLists.txt} qui contient tout les information concernant les fichier
du project.

\subsubsection{Exécution avec \texttt{N = 4} et \texttt{P = 3}}
Pour exécuter \texttt{P} processus on a écrit un programme \texttt{run\_clients}
qui prends \texttt{N} comme  paramètre du \texttt{main} et qui \texttt{N}
processus clients en utlisant les fonctions \texttt{fork, execlp}.

Voici les sorties de l'exécution des processus :

\lstinputlisting[caption=Voiture.c, style=Bash]{../question1out/voiture.out}
\lstinputlisting[caption=run\_clients.c, style=Bash]{../question1out/run_clients.out}


