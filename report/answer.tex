\subsection{Question 1}
\subsubsection{Implementaion}
Pour l'implémentation du limite des trounée pour chaque clients, on doit
sauvegarder le nombre du tournée pour chaque clients. Pour cela on a définie les
variables \texttt{int nbClients}, \texttt{int clients[]} et \texttt{int
  tours[]} dans la structure \texttt{state}.
\texttt{nbClients} contients le nombre de clients dans le systèmes.
\texttt{clients} contients les \textbf{pids} des clients dans le système.
\texttt{tours} contients le nombre de tournées pour chaque \textbf{pid} dans
\texttt{clients}.

on définie aussi les variables \texttt{MAX_TOURS} qui peut être mise à jourer
par le processus voiture et \texttt{MAX_CLIENTS} qui contient le nombre maximum
du clients qui peut être enregister dans les tableaux précedents.

Une sémahpore \texttt{mutex3} pour protéger les variables \texttt{int nbClients}, \texttt{int clients[]} et \texttt{int
  tours[]}

\textbf{Note :} une milleur solution peut etre implementer en utilisant les structure
dynamiques.

L'idée est que chaque nouveau clients (son \textbf{pid}) est sauvegarder dans la
case numéro \texttt{nbClients} du tableau \textbf{clients} et son nombre de
tournée associer est sauvegardé dans la case numéro \texttt{nbClients} du
tableau \texttt{tours}.

A l'arrivé du clients il fait appele à la foction \texttt{inscription} qui lui
enregister dans le tableau \texttt{clients} et initialise son nombre de tours à zero.

Pour pouvoir fait une tournée (la condition d'entrée) il fait appele à la
fonction  \texttt{peutTourner} qui retourne \texttt{vrai} que si le nombre de
tours du clients est inférieur au nombre maximale de trournée

Chaque fin de tournées le processus cliens fait appele à la fonction
\texttt{finTour} qui incrémente son nombre de tours.

\subsubsection{Exécution avec \texttt{N = 4} et \texttt{P = 3}}
Pour exécuter \texttt{P} processus on a écrit un programme \texttt{run\_clients}
qui prends \texttt{N} comme  paramètre du \texttt{main} et qui crée \texttt{N}
processus clients en utlisant les fonctions \texttt{fork, execlp}.

Voici les sorties de l'exécution des processus :

%\lstinputlisting[caption=Voiture.c, style=Bash]{../out/voiture.out}
%\lstinputlisting[caption=run\_clients.c, style=Bash]{../out/run_clients.out}


