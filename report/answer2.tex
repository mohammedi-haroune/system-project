<<<<<<< HEAD
\subsection{Question 1}
\subsubsection{Implementaion}
Pour l'implémentation du problème donnée en utilisant le système \texttt{IPC V},
on a définit les sémaphores indiqué dans la solution théorique donnée à l'aide
des appels système  linux \texttt{semget, semctl} et une mémoire partagée pour
sauvegarder le nombre de clients embarquées et débarquées (la structure
\texttt{state}). les mémoires partagées sont gérées à l'aide des appels système 
\texttt{shmget, shmat}. Notant que l'utlisation de ces appels système nécessite la création d'un clé \texttt{key\_t}.

Pour facilité la tache de compilation des différents fichiers \textbf{C} on a
utlisé l'outil \textbf{CMake}. pour cela on écrit le fichier
\texttt{CMakeLists.txt} qui contient tout les information concernant les fichier
du project.

\subsubsection{Exécution avec \texttt{N = 4} et \texttt{P = 3}}
Pour exécuter \texttt{P} processus on a écrit un programme \texttt{run\_clients}
qui prends \texttt{N} comme  paramètre du \texttt{main} et qui \texttt{N}
processus clients en utlisant les fonctions \texttt{fork, execlp}.

Voici les sorties de l'exécution des processus :

\lstinputlisting[caption=Voiture.c, style=Bash]{../question1out/voiture.out}
\lstinputlisting[caption=run\_clients.c, style=Bash]{../question1out/run_clients.out}

=======
\subsection{Question 2}
l'envoie un signal kill a un des processus client pendant que la
voiture est en tournée (commande: \texttt{ kill -9 pid}) bloque le processuss
voiture et tout autres processus clients dans le systems. le
blockage est dû au vaiable \texttt{nbDebarques} qui ne sera jamais égales à
\texttt{P} et donc le processus voiture se block dans l'instruction
\texttt{P(semid, semTousDehors)} et les processus clients se block dans
l'instruction \texttt{P(semid, semEmbarquement)}.

le flag \texttt{SEM\_UNDO} permet au processus de libérer les sémaphores dans le
quelle il a l'acces avant sa terminision  (remetre en cause les
différentes modifications sur les différentes sémaphores) en utilisant une variable special
\texttt{semadj} qui est définie pour chaque processus et pour chaque semaphore
et qui sauvgarde la somme de toutes les operations effectuer par chaque
processus sur chaque sémaphore.

Avec l'utlisation de ce flag dans les opérations des sémaphores
\texttt{P} et \texttt{V} le problème ne sera pas réglé puisque il été
engendré par la valeur du variable \texttt{nbDebarques} non pas par des valuers
de sémaphores.

On peut réglé le problème si on utilise un sémaphore \texttt{semDedans} qui sera
incréménté a chaque embarquement d'un processus et décrementé à chaque
débarquement, comme ça si on tue un processus qui a monté et qui n'a pas encore
decendu (c'est la cause du problème précedent) la valeur du sémaphore
\texttt{semDedans} va décrementer en specifiant le flag \texttt{SEM\_UNDO}, et
donc le processus voiture ne va pas bloquer puisque le nombre des clients dans
la voiture 

\textbf{Détail sur \texttt{SEM\_UNDO}}

\begin{itemize}
\item
  si la valeur du sémaphore est modifié par \texttt{SETVALL} ou \texttt{SETALL} la
  valeur du variable special \texttt{semadj} est reinitialisé. On doit pensé à
  utilisé les flags du système \texttt{CLONE\_SYSTEMV} dans ce cas.\\

\item
  l'OS linux applique la stratégie \textbf{décrementer la valeur tant que possible} lors du
  remetre en cause des operations effectuer par un processus sur un sémaphore
  (effet du flag \texttt{SEM\_UNDO}).
\end{itemize}
>>>>>>> parc/question2

